\documentclass[]{article}
\usepackage{lmodern}
\usepackage{amssymb,amsmath}
\usepackage{ifxetex,ifluatex}
\usepackage{fixltx2e} % provides \textsubscript
\ifnum 0\ifxetex 1\fi\ifluatex 1\fi=0 % if pdftex
  \usepackage[T1]{fontenc}
  \usepackage[utf8]{inputenc}
\else % if luatex or xelatex
  \ifxetex
    \usepackage{mathspec}
  \else
    \usepackage{fontspec}
  \fi
  \defaultfontfeatures{Mapping=tex-text,Scale=MatchLowercase}
  \newcommand{\euro}{€}
\fi
% use upquote if available, for straight quotes in verbatim environments
\IfFileExists{upquote.sty}{\usepackage{upquote}}{}
% use microtype if available
\IfFileExists{microtype.sty}{%
\usepackage{microtype}
\UseMicrotypeSet[protrusion]{basicmath} % disable protrusion for tt fonts
}{}
\makeatletter
\@ifpackageloaded{hyperref}{}{%
\ifxetex
  \usepackage[setpagesize=false, % page size defined by xetex
              unicode=false, % unicode breaks when used with xetex
              xetex]{hyperref}
\else
  \usepackage[unicode=true]{hyperref}
\fi
}
\@ifpackageloaded{color}{
    \PassOptionsToPackage{usenames,dvipsnames}{color}
}{%
    \usepackage[usenames,dvipsnames]{color}
}
\makeatother
\hypersetup{breaklinks=true,
            bookmarks=true,
            pdfauthor={},
            pdftitle={},
            colorlinks=true,
            citecolor=blue,
            urlcolor=blue,
            linkcolor=magenta,
            pdfborder={0 0 0}
            }
\urlstyle{same}  % don't use monospace font for urls
\setlength{\parindent}{0pt}
\setlength{\parskip}{6pt plus 2pt minus 1pt}
\setlength{\emergencystretch}{3em}  % prevent overfull lines
\providecommand{\tightlist}{%
  \setlength{\itemsep}{0pt}\setlength{\parskip}{0pt}}
\setcounter{secnumdepth}{0}

\date{}

% Redefines (sub)paragraphs to behave more like sections
\ifx\paragraph\undefined\else
\let\oldparagraph\paragraph
\renewcommand{\paragraph}[1]{\oldparagraph{#1}\mbox{}}
\fi
\ifx\subparagraph\undefined\else
\let\oldsubparagraph\subparagraph
\renewcommand{\subparagraph}[1]{\oldsubparagraph{#1}\mbox{}}
\fi

\begin{document}

\section{Matheus Albuquerque Brasil}\label{matheus-albuquerque-brasil}

\textbf{Localidade:} Fortaleza, CE - Brazil

\textbf{Blog:} \href{http://matheusbrasil.com}{matheusbrasil.com}

\textbf{E-mail:} matheus.brasil10@gmail.com

\textbf{Github:} \href{https://github.com/mabrasil}{mabrasil}

\textbf{LinkedIn:}
\href{https://www.linkedin.com/in/matheusalbuquerque}{Matheus
Albuquerque}

\textbf{Medium:} \href{https://medium.com/@matheusalbuquerque}{Matheus
Albuquerque}

\subsection{Sobre}\label{sobre}

Eu sou um desenvolvedor Full-Stack que vive em
\href{http://pt.wikipedia.org/wiki/Fortaleza}{Fortaleza, CE}.

Comecei a me interessar por programação aos 12 anos - não muito tempo
atrás - e, desde então, estudei/experimentei um número considerável de
linguagens de programação, indo desde as mais conhecidas, como Java,
C\#, Ruby, Python, Brainfuck - sim, essa é muito conhecida -, e
JavaScript, passando por voltadas ao paradigma funcional - como Haskell,
Elixir, Erlang e Clojure - até algumas esotéricas, como Shakespeare,
LOLCODE, 4Lang, ZOMBIE etc.

Eu me considero um \emph{alquimista da web} que está sempre testando
novas tecnologias. Atualmente, tento ser o mais ativo o possível em
comunidades locais como
\href{https://www.facebook.com/groups/fortalezadevelopers/}{Dev I/O
Fortaleza}, \href{http://ionicbrazil.com/}{Ionic Brazil},
\href{https://github.com/lambda-io}{The Lambda I/O Foundation},
\href{http://js4girls-fortaleza.github.io/}{JS4Girls Fortaleza} entre
outras.

Minhas paixões incluem o design e implementação de projetos Full Stack,
projetos de código aberto - e estilos como indie, rock e eletrônica. Eu
também sou aluno do curso técnico em informática do Instituto Federal do
Ceará.

\subsection{Habilidades}\label{habilidades}

\subsubsection{Pessoais}\label{pessoais}

\begin{itemize}
\tightlist
\item
  Foco em resolver problemas
\item
  Proatividade
\item
  Amor por código aberto
\item
  Amor por compartilhar/disseminar tecnologias
\item
  Facilidade ao falar em público
\item
  Trabalhar em equipe
\item
  Saber gerenciar o tempo
\item
  Habilidade de aceitar e aprender com críticas
\item
  Flixibilidade/Fácil adaptação
\end{itemize}

\subsubsection{Técnicas}\label{tuxe9cnicas}

\paragraph{Desenvolvimento e Instrumentação
Front-End}\label{desenvolvimento-e-instrumentauxe7uxe3o-front-end}

\begin{itemize}
\item
  Linguagens de Marcação
\item
  HTML(5) e suas APIs
\item
  Templating Engines JavaScript

  \begin{quote}
  Jade, Nunjucks, EJS.
  \end{quote}
\item
  Estilização
\item
  CSS(3)
\item
  Pré-processadores CSS

  \begin{quote}
  Stylus, Sass, Less.
  \end{quote}
\item
  Frameworks CSS

  \begin{quote}
  Bootstrap, Foundation.
  \end{quote}
\item
  Design Responsivo (Mobile-First)
\item
  Automação
\end{itemize}

\begin{quote}
Grunt, Gulp, NPM (para automação), Yeoman.
\end{quote}

\begin{itemize}
\tightlist
\item
  Frameworks JavaScript
\end{itemize}

\begin{quote}
AngularJS, EmberJS.
\end{quote}

\begin{itemize}
\tightlist
\item
  Bibliotecas JavaScript
\end{itemize}

\begin{quote}
jQuery, zepto, ReactJS etc.
\end{quote}

\begin{itemize}
\tightlist
\item
  Pré-processadores JavaScript
\end{itemize}

\begin{quote}
CoffeeScript, LiveScript, TypeScript.
\end{quote}

\begin{itemize}
\tightlist
\item
  Desenvolvimento Mobile
\end{itemize}

\begin{quote}
Aplicações Híbridas com Ionic e React Native.
\end{quote}

\paragraph{Desenvolvimento Back-End}\label{desenvolvimento-back-end}

\begin{itemize}
\tightlist
\item
  Node.js e seus frameworks
\end{itemize}

\begin{quote}
express.js, koa.js, meteor.js, Derby, ReactJS etc.
\end{quote}

\paragraph{Bancos de Dados}\label{bancos-de-dados}

\begin{itemize}
\tightlist
\item
  SGBDs
\end{itemize}

\begin{quote}
MySQL, Postgres.
\end{quote}

\begin{itemize}
\tightlist
\item
  NoSQL
\end{itemize}

\begin{quote}
MongoDB, LevelDB.
\end{quote}

\paragraph{Versionamento de Software}\label{versionamento-de-software}

\begin{quote}
Git e social coding via Github
\end{quote}

\paragraph{Teste de Sofware e Devops}\label{teste-de-sofware-e-devops}

\begin{itemize}
\item
  TDD \& BDD
\item
  Testes unitários em JavaScript
\item
  Testes E2E \& A/B em front-end
\item
  Integração Contínua
\end{itemize}

\subsubsection{Idiomas}\label{idiomas}

\begin{itemize}
\tightlist
\item
  Português
\end{itemize}

\begin{quote}
Nível nativo.
\end{quote}

\begin{itemize}
\tightlist
\item
  English
\end{itemize}

\begin{quote}
Nível avançado de proficiência.
\end{quote}

\subsection{Educação}\label{educauxe7uxe3o}

\subsubsection{Cursos em Andamento}\label{cursos-em-andamento}

\textbf{Técnico em Informática} @ \emph{Instituto Federal de Educação,
Ciência e Tecnologia do Ceará}

\emph{2013} - \emph{2017}

\begin{quote}
Aluno do curso técnico integrado ao Ensino Médio na área de Informática
(TII) que forma profissionais para atuar diretamente com sistemas de
informática, abrangendo a utilização eficiente, a programação e o
suporte ao uso de equipamentos.
\end{quote}

\textbf{Curso Básico de Língua Inglesa} @ \emph{Casa de Cultura
Britânica (CCB) da Universidade Federal do Ceará (UFC)}

\emph{2014} - \emph{2016}

\begin{quote}
Aluno do curso da Casa de Cultura Britânica (CCB) que desenvolve de
forma integrada as habilidades de ouvir, falar, ler e escrever em nível
básico.
\end{quote}

\subsubsection{Cursos Concluídos}\label{cursos-concluuxeddos}

\textbf{Teens Level Course} @ \emph{Instituto Brasil-Estados Unidos no
Ceará (IBEU-CE)}

\emph{2010} - \emph{2012}

\begin{quote}
Aluno do curso do Instituto Brasil-Estados Unidos no Ceará (IBEU-CE) que
desenvolve de forma integrada as habilidades de ouvir, falar, ler e
escrever em nível básico.
\end{quote}

\textbf{Curso Introdutório de Robótica} @ \emph{Instituto Brasileiro de
Educação Profissional (IBEPRO)}

\emph{2012} - \emph{2012}

\begin{quote}
Aluno em um curso básico de Robótica, que introduz muitos conceitos
fundamentais da eletrônica digital.
\end{quote}

\subsection{Reconhecimentos e
Premiações}\label{reconhecimentos-e-premiauxe7uxf5es}

\textbf{2º Colocado} @ \emph{III Startup Weekend Fortaleza}

\begin{quote}
Trabalhei como desenvolvedor front-end/mobile no projeto \emph{Chego Lá}
que consiste em uma solução de monitoramento de rendimento escolar que
ficou em 2º lugar na
\href{http://www.up.co/communities/brazil/fortaleza/startup-weekend/4487}{III
Startup Weekend Fortaleza}.
\end{quote}

\textbf{2º Colocado} @ \emph{I Feira de Hardware \& Software (FHS) do
Instituto Federal de Educação, Ciência e Tecnologia do Ceará (IFCE)}

\begin{quote}
Trabalhei como desenvolvedor full-stack no projeto \emph{WebCampi}, que
consiste em uma inovadora rede social de compartilhamento de conteúdo
acadêmico que ficou em 2º lugar na categoria \emph{Software} em uma
feira interna do Instituto Federal de Educação, Ciência e Tecnologia do
Ceará (IFCE).
\end{quote}

\textbf{Top Student} @ \emph{Instituto Brasil-Estados Unidos no Ceará
(IBEU-CE)}

\begin{quote}
Reconhecimento pelo bom desempenho apresentado ao longo do Teens Course
no IBEU-SEDE (2012/2).
\end{quote}

\subsection{Projetos}\label{projetos}

Alguns dos projetos em que eu já trabalhei.

\paragraph{\texorpdfstring{\href{https://github.com/devevents/conf-app-boilerplate}{Conf
App Boilerplate}}{Conf App Boilerplate}}\label{conf-app-boilerplate}

\begin{quote}
Base de código para criação de uma aplicação móvel para ajudar pessoas
que precisam organizar conferências/eventos e não têm tempo suficiente
para criar tal aplicação com informações do evento.
\end{quote}

\paragraph{\texorpdfstring{\href{https://play.google.com/store/apps/details?id=com.devevents.frontinfortaleza}{Front
in Fortaleza}}{Front in Fortaleza}}\label{front-in-fortaleza}

\begin{quote}
Um aplicativo móvel com todos os tipos de informações sobre o maior
evento front-end no Ceará, incluindo o local, palestrantes, programação,
patrocinadores, parceiros etc.
\end{quote}

\paragraph{\texorpdfstring{\href{https://github.com/mabrasil/lumberpack}{Lumberpack}}{Lumberpack}}\label{lumberpack}

\begin{quote}
Base de código para facilitar a criação de projetos web com uma
variedade de tecnologias atuais, como Gulp, Jade, Stylus, Livescript
etc.
\end{quote}

\paragraph{\texorpdfstring{\href{https://github.com/mabrasil/conf-boilerplate}{Conf
Boilerplate}}{Conf Boilerplate}}\label{conf-boilerplate}

\begin{quote}
Fork do projeto original da BrazilJS, o
\href{https://github.com/braziljs/conf-boilerplate}{Conf Boilerplate},
que visa a ajudar pessoas que precisam organizar conferências/eventos e
não têm tempo suficiente para criar um site para este, no qual eu fiz
uma completa reformulação na arquitetura.
\end{quote}

\paragraph{\texorpdfstring{\href{https://github.com/mabrasil/xzibit}{xzibit}}{xzibit}}\label{xzibit}

\begin{quote}
Base de código para facilitar a criação de apresentações de slides com
uma variedade de tecnologias atuais, como Gulp, Jade, Stylus etc.
\end{quote}

\paragraph{\texorpdfstring{\href{https://github.com/mabrasil/codeicons}{Code
Icons}}{Code Icons}}\label{code-icons}

\begin{quote}
Conjunto de ícones de 138 linguagens de programação, frameworks e outras
tecnologias.
\end{quote}

\paragraph{\texorpdfstring{\href{https://github.com/mabrasil/milla-theme}{Milla
Theme}}{Milla Theme}}\label{milla-theme}

\begin{quote}
Um simples \emph{(e alternativo)} tema para SublimeText, Textmate e
outros editores de texto que usa uma paleta de cores flat.
\end{quote}

\subsection{Extras}\label{extras}

\subsubsection{Palestrante em:}\label{palestrante-em}

\paragraph{\texorpdfstring{\href{http://flisolce.org/}{FLISOL
2015}}{FLISOL 2015}}\label{flisol-2015}

\begin{itemize}
\tightlist
\item
  \textbf{Título:} \emph{Uma Introdução a Git \& Github}
\item
  \textbf{Quando:} 25 de Abril de 2015
\item
  \textbf{Onde:} Instituto Federal de Educação, Ciência e Tecnologia do
  Ceará, Campus Fortaleza.
\item
  \textbf{Presentes:} ≈60
\end{itemize}

\paragraph{\texorpdfstring{\href{http://www.cocoaheads.com.br/agendas/detalhes/79/}{3º
CocoaHeads
Fortaleza}}{3º CocoaHeads Fortaleza}}\label{uxba-cocoaheads-fortaleza}

\begin{itemize}
\tightlist
\item
  \textbf{Título:} \emph{Aplicações Híbridas com Ionic}
\item
  \textbf{Quando:} 28 de Maio de 2015
\item
  \textbf{Onde:} Instituto Federal de Educação, Ciência e Tecnologia do
  Ceará, Campus Fortaleza.
\item
  \textbf{Presentes:} ≈40
\end{itemize}

\paragraph{\texorpdfstring{\href{http://pylestras.org/evento/ix-pylestras/}{IX
Pylestras}}{IX Pylestras}}\label{ix-pylestras}

\begin{itemize}
\tightlist
\item
  \textbf{Título:} \emph{Aventurando-se com metadados: Micro Data e Open
  Graph}
\item
  \textbf{Quando:} 30 de Maio de 2015
\item
  \textbf{Onde:} Universidade de Fortaleza - Unifor.
\item
  \textbf{Presentes:} ≈40
\end{itemize}

\paragraph{\texorpdfstring{\href{http://www.meetup.com/pt/Ionic-Ceara/events/224620543/}{I
Ionic Meetup
Fortaleza}}{I Ionic Meetup Fortaleza}}\label{i-ionic-meetup-fortaleza}

\begin{itemize}
\tightlist
\item
  \textbf{Título:} \emph{Um projeto mobile open source em duas semanas
  (ou uma talk sobre como melhorar seu workflow Ionic)}
\item
  \textbf{Quando:} 15 de Agosto de 2015
\item
  \textbf{Onde:} Universidade de Fortaleza - Unifor.
\item
  \textbf{Presentes:} ≈40
\end{itemize}

\paragraph{\texorpdfstring{\href{http://unidevce.github.io/}{Unidev
2}}{Unidev 2}}\label{unidev-2}

\begin{itemize}
\tightlist
\item
  \textbf{Título:} \emph{De volta para o futuro: funcional nos dias
  atuais}
\item
  \textbf{Quando:} 26 de Setembro de 2015
\item
  \textbf{Onde:} Universidade de Fortaleza - Unifor.
\item
  \textbf{Presentes:} ≈50
\end{itemize}

\paragraph{\texorpdfstring{\href{http://www.seti.ufc.br/}{SETIC
2015}}{SETIC 2015}}\label{setic-2015}

\begin{itemize}
\tightlist
\item
  \textbf{Título:} \emph{De volta para o futuro: funcional nos dias
  atuais}
\item
  \textbf{Quando:} 20 de Outubro de 2015
\item
  \textbf{Onde:} Universidade Federal do Ceará - Campus do Pici.
\item
  \textbf{Presentes:} ≈30
\end{itemize}

\subsubsection{Organizador em:}\label{organizador-em}

\begin{itemize}
\tightlist
\item
  \href{http://ionicbrazil.com/}{Ionic Meetup Fortaleza}
\item
  \href{https://github.com/lambda-io}{Lambda I/O Foundation}
\item
  \href{http://js4girls-fortaleza.github.io/}{JS4Girls Fortaleza}
\end{itemize}

\end{document}
