\documentclass[]{article}
\usepackage{lmodern}
\usepackage{amssymb,amsmath}
\usepackage{ifxetex,ifluatex}
\usepackage{fixltx2e} % provides \textsubscript
\ifnum 0\ifxetex 1\fi\ifluatex 1\fi=0 % if pdftex
  \usepackage[T1]{fontenc}
  \usepackage[utf8]{inputenc}
\else % if luatex or xelatex
  \ifxetex
    \usepackage{mathspec}
  \else
    \usepackage{fontspec}
  \fi
  \defaultfontfeatures{Mapping=tex-text,Scale=MatchLowercase}
  \newcommand{\euro}{€}
\fi
% use upquote if available, for straight quotes in verbatim environments
\IfFileExists{upquote.sty}{\usepackage{upquote}}{}
% use microtype if available
\IfFileExists{microtype.sty}{%
\usepackage{microtype}
\UseMicrotypeSet[protrusion]{basicmath} % disable protrusion for tt fonts
}{}
\makeatletter
\@ifpackageloaded{hyperref}{}{%
\ifxetex
  \usepackage[setpagesize=false, % page size defined by xetex
              unicode=false, % unicode breaks when used with xetex
              xetex]{hyperref}
\else
  \usepackage[unicode=true]{hyperref}
\fi
}
\@ifpackageloaded{color}{
    \PassOptionsToPackage{usenames,dvipsnames}{color}
}{%
    \usepackage[usenames,dvipsnames]{color}
}
\makeatother
\hypersetup{breaklinks=true,
            bookmarks=true,
            pdfauthor={},
            pdftitle={},
            colorlinks=true,
            citecolor=blue,
            urlcolor=blue,
            linkcolor=magenta,
            pdfborder={0 0 0}
            }
\urlstyle{same}  % don't use monospace font for urls
\setlength{\parindent}{0pt}
\setlength{\parskip}{6pt plus 2pt minus 1pt}
\setlength{\emergencystretch}{3em}  % prevent overfull lines
\providecommand{\tightlist}{%
  \setlength{\itemsep}{0pt}\setlength{\parskip}{0pt}}
\setcounter{secnumdepth}{0}

\date{}

% Redefines (sub)paragraphs to behave more like sections
\ifx\paragraph\undefined\else
\let\oldparagraph\paragraph
\renewcommand{\paragraph}[1]{\oldparagraph{#1}\mbox{}}
\fi
\ifx\subparagraph\undefined\else
\let\oldsubparagraph\subparagraph
\renewcommand{\subparagraph}[1]{\oldsubparagraph{#1}\mbox{}}
\fi

\begin{document}

\section{Matheus Albuquerque Brasil}\label{matheus-albuquerque-brasil}

\textbf{Localidade:} Fortaleza, CE - Brazil

\textbf{Telefone}: +55 (85) 99644-1530

\textbf{Blog:} \href{http://matheusbrasil.com}{matheusbrasil.com}

\textbf{E-mail:} matheus.brasil10@gmail.com

\textbf{Github:} \href{https://github.com/mabrasil}{mabrasil}

\textbf{LinkedIn:}
\href{https://www.linkedin.com/in/matheusalbuquerque}{Matheus
Albuquerque}

\subsection{Sobre}\label{sobre}

Desenvolvedor Full-Stack que vive em
\href{http://pt.wikipedia.org/wiki/Fortaleza}{Fortaleza, CE}.

Comecei a me interessar por programação aos 12 anos e, desde então,
estudei/experimentei um número considerável de linguagens de
programação, indo desde as mais conhecidas, passando por voltadas ao
paradigma funcional até algumas esotéricas.

Eu me considero um \emph{alquimista da web} que está sempre testando
novas tecnologias. Atualmente, sou aluno do curso técnico em informática
do Instituto Federal do Ceará (IFCE) e tento ser o mais ativo o possível
em comunidades locais como
\href{https://www.facebook.com/groups/fortalezadevelopers/}{Dev I/O
Fortaleza}, \href{http://ionicbrazil.com/}{Ionic Brazil},
\href{https://github.com/lambda-io}{The Lambda I/O Foundation},
\href{http://js4girls-fortaleza.github.io/}{JS4Girls Fortaleza} entre
outras.

Minhas paixões incluem o design e implementação de projetos Full Stack,
projetos de código aberto e comunidades de tecnologias.

\subsection{Habilidades}\label{habilidades}

\subsubsection{Pessoais}\label{pessoais}

\begin{itemize}
\tightlist
\item
  Foco em resolver problemas
\item
  Proatividade
\item
  Amor por código aberto
\item
  Amor por compartilhar/disseminar tecnologias
\item
  Facilidade ao falar em público
\item
  Trabalhar em equipe
\item
  Saber gerenciar o tempo
\item
  Habilidade de aceitar e aprender com críticas
\item
  Flixibilidade/Fácil adaptação
\end{itemize}

\subsubsection{Técnicas}\label{tuxe9cnicas}

\begin{itemize}
\tightlist
\item
  Desenvolvimento e Instrumentação Front-End
\end{itemize}

\begin{quote}
Linguagens de Marcação, Estilização, Design Responsivo (Mobile-First),
Automação, Frameworks JavaScript, Bibliotecas JavaScript,
Pré-processadores JavaScript.
\end{quote}

\begin{itemize}
\tightlist
\item
  Desenvolvimento Mobile
\end{itemize}

\begin{quote}
Aplicações Híbridas baseadas em tecnologias web.
\end{quote}

\begin{itemize}
\tightlist
\item
  Back-End Development
\end{itemize}

\begin{quote}
Node.js e seus frameworks.
\end{quote}

\begin{itemize}
\tightlist
\item
  Bancos de Dados
\end{itemize}

\begin{quote}
SGBDs e NoSQL.
\end{quote}

\begin{itemize}
\tightlist
\item
  Versionamento de Software
\end{itemize}

\begin{quote}
Git e social coding via Github.
\end{quote}

\begin{itemize}
\tightlist
\item
  Teste de Sofware e Devops
\end{itemize}

\begin{quote}
TDD \& BDD, Testes unitários em JavaScript, Testes E2E \& A/B em
front-end, Integração Contínua.
\end{quote}

\subsubsection{Idiomas}\label{idiomas}

\begin{itemize}
\item
  Português (Nativo)
\item
  English (Nível avançado de proficiência)
\end{itemize}

\subsection{Educação}\label{educauxe7uxe3o}

\subsubsection{Cursos em Andamento}\label{cursos-em-andamento}

\begin{itemize}
\item
  \textbf{Técnico em Informática} @ \emph{Instituto Federal de Educação,
  Ciência e Tecnologia do Ceará}
\item
  \textbf{Curso Básico de Língua Inglesa} @ \emph{Casa de Cultura
  Britânica (CCB) da Universidade Federal do Ceará (UFC)}
\end{itemize}

\subsubsection{Cursos Concluídos}\label{cursos-concluuxeddos}

\begin{itemize}
\item
  \textbf{Teens Level Course} @ \emph{Instituto Brasil-Estados Unidos no
  Ceará (IBEU-CE)}
\item
  \textbf{Curso Introdutório de Robótica} @ \emph{Instituto Brasileiro
  de Educação Profissional (IBEPRO)}
\end{itemize}

\subsection{Reconhecimentos e
Premiações}\label{reconhecimentos-e-premiauxe7uxf5es}

\begin{itemize}
\item
  \textbf{2º Colocado} @ \emph{III Startup Weekend Fortaleza}
\item
  \textbf{2º Colocado} @ \emph{I Feira de Hardware \& Software (FHS) do
  Instituto Federal de Educação, Ciência e Tecnologia do Ceará (IFCE)}
\item
  \textbf{Top Student} @ \emph{Instituto Brasil-Estados Unidos no Ceará
  (IBEU-CE)}
\end{itemize}

\subsection{Projetos}\label{projetos}

Alguns dos projetos em que eu já trabalhei.

\begin{itemize}
\tightlist
\item
  \href{https://github.com/devevents/conf-app-boilerplate}{Conf App
  Boilerplate}
\item
  \href{https://play.google.com/store/apps/details?id=com.devevents.frontinfortaleza}{Front
  in Fortaleza}
\item
  \href{https://github.com/mabrasil/lumberpack}{Lumberpack}
\item
  \href{https://github.com/mabrasil/conf-boilerplate}{Conf Boilerplate}
\item
  \href{https://github.com/mabrasil/xzibit}{xzibit}
\item
  \href{https://github.com/mabrasil/codeicons}{Code Icons}
\item
  \href{https://github.com/mabrasil/milla-theme}{Milla Theme}
\end{itemize}

\subsection{Extras}\label{extras}

\subsubsection{Palestrante em:}\label{palestrante-em}

\begin{itemize}
\tightlist
\item
  \href{http://flisolce.org/}{FLISOL 2015}
\item
  \href{http://www.cocoaheads.com.br/agendas/detalhes/79/}{3º CocoaHeads
  Fortaleza}
\item
  \href{http://pylestras.org/evento/ix-pylestras/}{IX Pylestras}
\item
  \href{http://www.meetup.com/pt/Ionic-Ceara/events/224620543/}{I Ionic
  Meetup Fortaleza}
\item
  \href{http://unidevce.github.io/}{Unidev 2}
\item
  \href{http://www.seti.ufc.br/}{SETIC 2015}
\end{itemize}

\subsubsection{Organizador em:}\label{organizador-em}

\begin{itemize}
\tightlist
\item
  \href{http://ionicbrazil.com/}{Ionic Meetup Fortaleza}
\item
  \href{https://github.com/lambda-io}{Lambda I/O Foundation}
\item
  \href{http://js4girls-fortaleza.github.io/}{JS4Girls Fortaleza}
\end{itemize}

\end{document}
