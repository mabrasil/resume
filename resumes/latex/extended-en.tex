\documentclass[]{article}
\usepackage{lmodern}
\usepackage{amssymb,amsmath}
\usepackage{ifxetex,ifluatex}
\usepackage{fixltx2e} % provides \textsubscript
\ifnum 0\ifxetex 1\fi\ifluatex 1\fi=0 % if pdftex
  \usepackage[T1]{fontenc}
  \usepackage[utf8]{inputenc}
\else % if luatex or xelatex
  \ifxetex
    \usepackage{mathspec}
  \else
    \usepackage{fontspec}
  \fi
  \defaultfontfeatures{Mapping=tex-text,Scale=MatchLowercase}
  \newcommand{\euro}{€}
\fi
% use upquote if available, for straight quotes in verbatim environments
\IfFileExists{upquote.sty}{\usepackage{upquote}}{}
% use microtype if available
\IfFileExists{microtype.sty}{%
\usepackage{microtype}
\UseMicrotypeSet[protrusion]{basicmath} % disable protrusion for tt fonts
}{}
\makeatletter
\@ifpackageloaded{hyperref}{}{%
\ifxetex
  \usepackage[setpagesize=false, % page size defined by xetex
              unicode=false, % unicode breaks when used with xetex
              xetex]{hyperref}
\else
  \usepackage[unicode=true]{hyperref}
\fi
}
\@ifpackageloaded{color}{
    \PassOptionsToPackage{usenames,dvipsnames}{color}
}{%
    \usepackage[usenames,dvipsnames]{color}
}
\makeatother
\hypersetup{breaklinks=true,
            bookmarks=true,
            pdfauthor={},
            pdftitle={},
            colorlinks=true,
            citecolor=blue,
            urlcolor=blue,
            linkcolor=magenta,
            pdfborder={0 0 0}
            }
\urlstyle{same}  % don't use monospace font for urls
\setlength{\parindent}{0pt}
\setlength{\parskip}{6pt plus 2pt minus 1pt}
\setlength{\emergencystretch}{3em}  % prevent overfull lines
\providecommand{\tightlist}{%
  \setlength{\itemsep}{0pt}\setlength{\parskip}{0pt}}
\setcounter{secnumdepth}{0}

\date{}

% Redefines (sub)paragraphs to behave more like sections
\ifx\paragraph\undefined\else
\let\oldparagraph\paragraph
\renewcommand{\paragraph}[1]{\oldparagraph{#1}\mbox{}}
\fi
\ifx\subparagraph\undefined\else
\let\oldsubparagraph\subparagraph
\renewcommand{\subparagraph}[1]{\oldsubparagraph{#1}\mbox{}}
\fi

\begin{document}

\section{Matheus Albuquerque Brasil}\label{matheus-albuquerque-brasil}

\textbf{Location:} Fortaleza, CE - Brazil

\textbf{Phone Number}: +55 (85) 99644-1530

\textbf{Blog:} \href{http://matheusbrasil.com}{matheusbrasil.com}

\textbf{E-mail:} matheus.brasil10@gmail.com

\textbf{Github:} \href{https://github.com/mabrasil}{mabrasil}

\textbf{LinkedIn:}
\href{https://www.linkedin.com/in/matheusalbuquerque}{Matheus
Albuquerque}

\textbf{Medium:} \href{https://medium.com/@matheusalbuquerque}{Matheus
Albuquerque}

\subsection{About Me}\label{about-me}

I am a full stack developer who lives in
\href{http://pt.wikipedia.org/wiki/Fortaleza}{Fortaleza, CE}.

I have started playing with programming since four years ago and I have
played with some languages; from mainstream ones like Java, C\#, Ruby,
Python and JavaScript, going through functional ones like Haskell,
Elixir, Erlang and Clojure to really esoteric ones like Shakespeare,
LOLCODE, 4Lang, ZOMBIE etc.

I consider myself a \emph{web alchemist} who is always trying new
technologies and specially playing with cool stuff. Nowadays I study IT
at Instituto Federal do Ceará and try to be as active as possible at
local communities like
\href{https://www.facebook.com/groups/fortalezadevelopers/}{Dev I/O
Fortaleza}, \href{http://ionicbrazil.com/}{Ionic Brazil},
\href{https://github.com/lambda-io}{The Lambda I/O Foundation},
\href{http://js4girls-fortaleza.github.io/}{JS4Girls Fortaleza} and
other ones.

My passions include the full stack design and implementation of
projects, open source stuff and programming communities.

\subsection{Skills}\label{skills}

\subsubsection{Soft Skills}\label{soft-skills}

\begin{itemize}
\tightlist
\item
  Problem-solving skills
\item
  Proactive attitude
\item
  Love for open source contributing
\item
  Love for spreading cool technologies
\item
  Public speaking
\item
  Acting as a team player who has leadership spirit
\item
  Time management abilities
\item
  Ability to accept and learn from criticism
\item
  Flexibility/Adaptability
\end{itemize}

\subsubsection{Technical Skills}\label{technical-skills}

\paragraph{Front-End Development \&
Tooling}\label{front-end-development-tooling}

\begin{itemize}
\item
  Markup
\item
  HTML(5) and its APIs
\item
  JS Templating Engines
\end{itemize}

\begin{quote}
Jade, Nunjucks, EJS.
\end{quote}

\begin{itemize}
\item
  Styles
\item
  CSS(3)
\item
  CSS Preprocessors
\end{itemize}

\begin{quote}
Stylus, Sass, Less.
\end{quote}

\begin{itemize}
\tightlist
\item
  CSS Frameworks
\end{itemize}

\begin{quote}
Bootstrap, Foundation.
\end{quote}

\begin{itemize}
\item
  Responsive Design (Mobile First)
\item
  Workflow Automation \& Scaffolding
\end{itemize}

\begin{quote}
Grunt, Gulp, NPM (as a build tool), Yeoman.
\end{quote}

\begin{itemize}
\tightlist
\item
  JavaScript Frameworks
\end{itemize}

\begin{quote}
AngularJS, EmberJS.
\end{quote}

\begin{itemize}
\tightlist
\item
  JavaScript Libraries
\end{itemize}

\begin{quote}
jQuery, zepto, ReactJS etc.
\end{quote}

\begin{itemize}
\tightlist
\item
  JavaScript Preprocessors
\end{itemize}

\begin{quote}
Coffeescript, Livescript, Typescript.
\end{quote}

\paragraph{Mobile Development}\label{mobile-development}

\begin{itemize}
\tightlist
\item
  Hybrid mobile apps
\end{itemize}

\begin{quote}
Ionic, React Native.
\end{quote}

\paragraph{Back-End Development}\label{back-end-development}

\begin{itemize}
\tightlist
\item
  Node.js and its frameworks
\end{itemize}

\begin{quote}
express.js, koa.js, meteor.js, Derby, ReactJS etc.
\end{quote}

\paragraph{Database Management
Systems}\label{database-management-systems}

\begin{itemize}
\tightlist
\item
  SGBDs
\end{itemize}

\begin{quote}
MySQL, Postgres.
\end{quote}

\begin{itemize}
\tightlist
\item
  NoSQL
\end{itemize}

\begin{quote}
MongoDB, LevelDB.
\end{quote}

\paragraph{Software Versioning}\label{software-versioning}

\begin{quote}
Git \& social coding via Github
\end{quote}

\paragraph{Software Testing \& Devops}\label{software-testing-devops}

\begin{itemize}
\item
  TDD \& BDD
\item
  JavaScript unit testing
\item
  E2E \& A/B front-end testing
\item
  Continous Integration
\end{itemize}

\subsubsection{Languages}\label{languages}

\begin{itemize}
\tightlist
\item
  Portuguese
\end{itemize}

\begin{quote}
Native speaker.
\end{quote}

\begin{itemize}
\tightlist
\item
  English
\end{itemize}

\begin{quote}
Advanced level of proficiency.
\end{quote}

\subsection{Education}\label{education}

\subsubsection{Ongoing Courses}\label{ongoing-courses}

\textbf{Computer Technician} @ \emph{Instituto Federal de Educação,
Ciência e Tecnologia do Ceará}

\emph{2013} - \emph{2017}

\begin{quote}
Student at a technical course in the area of Information Technology (IT)
integrated to the high school, which aims training professionals to work
directly with computer systems, including the efficient use, programming
and support.
\end{quote}

\textbf{English Language Basic Course} @ \emph{Casa de Cultura Britânica
(CCB) da Universidade Federal do Ceará (UFC)}

\emph{2014} - \emph{2016}

\begin{quote}
Student at a basic English course which develops skills in an integrated
way to hear, speak, read and write at a basic level (elementary,
pre-intermediate, intermediate).
\end{quote}

\subsubsection{Finished Courses}\label{finished-courses}

\textbf{Teens Level Course} @ \emph{Instituto Brasil-Estados Unidos no
Ceará (IBEU-CE)}

\emph{2010} - \emph{2012}

\begin{quote}
Student at a basic English course which develops skills in an integrated
way to hear, speak, read and write at a basic level (elementary,
pre-intermediate, intermediate).
\end{quote}

\textbf{Introductory Course in Robotics} @ \emph{Instituto Brasileiro de
Educação Profissional (IBEPRO)}

\emph{2012} - \emph{2012}

\begin{quote}
Student at a basic Robotics course which introduces many fundamental
concepts of digital electronics.
\end{quote}

\subsection{Recognitions and Awards}\label{recognitions-and-awards}

\textbf{2nd place} @ \emph{III Startup Weekend Fortaleza}

\begin{quote}
I worked as a front-end/mobile developer in the project \emph{Chego Lá}
which is a solution for monitoring school performance that scored 2nd
place in the
\href{http://www.up.co/communities/brazil/fortaleza/startup-weekend/4487}{III
Startup Weekend Fortaleza}.
\end{quote}

\textbf{2nd place} @ \emph{I Feira de Hardware \& Software (FHS) do
Instituto Federal de Educação, Ciência e Tecnologia do Ceará (IFCE)}

\begin{quote}
I worked as a full-stack developer in the project \emph{WebCampi} which
is an innovative academic content sharing network that scored 2nd place
in the Software category of an internal competition at Instituto Federal
de Educação, Ciência e Tecnologia do Ceará (IFCE).
\end{quote}

\textbf{Top Student} @ \emph{Instituto Brasil-Estados Unidos no Ceará
(IBEU-CE)}

\begin{quote}
Recognition for the outstanding result in the Teens Course at IBEU-SEDE
(2012/2).
\end{quote}

\subsection{Projects}\label{projects}

\begin{quote}
Some of the projects in which I've already worked.
\end{quote}

\paragraph{\texorpdfstring{\href{https://github.com/devevents/conf-app-boilerplate}{Conf
App Boilerplate}}{Conf App Boilerplate}}\label{conf-app-boilerplate}

\begin{quote}
Boilerplate of mobile application to help people who wants to organize
conferences/events and don't have time enough to create an app with
information about the event.
\end{quote}

\paragraph{\texorpdfstring{\href{https://play.google.com/store/apps/details?id=com.devevents.frontinfortaleza}{Front
in Fortaleza}}{Front in Fortaleza}}\label{front-in-fortaleza}

\begin{quote}
A mobile app with all kinds of information about the biggest front-end
event in Ceará, including the venue, speakers, schedule, sponsors,
partners etc.
\end{quote}

\paragraph{\texorpdfstring{\href{https://github.com/mabrasil/lumberpack}{Lumberpack}}{Lumberpack}}\label{lumberpack}

\begin{quote}
A simple boilerplate to easily bootstrap web projects with a bunch of
cool technologies, like Gulp, Jade, Stylus, Livescript etc.
\end{quote}

\paragraph{\texorpdfstring{\href{https://github.com/mabrasil/conf-boilerplate}{Conf
Boilerplate}}{Conf Boilerplate}}\label{conf-boilerplate}

\begin{quote}
Fork of the original project from BrazilJS
\href{https://github.com/braziljs/conf-boilerplate}{Conf Boilerplate} -
which aims \emph{``to help those people who wants to organize
conferences/events and don't have too much time to create the website of
it''} - on which I make a brand new tech stack.
\end{quote}

\paragraph{\texorpdfstring{\href{https://github.com/mabrasil/xzibit}{xzibit}}{xzibit}}\label{xzibit}

\begin{quote}
It's a boilerplate to allow you to easily create modern presentations
using Reveal.JS, Gulp, Jade, Stylus and more.
\end{quote}

\paragraph{\texorpdfstring{\href{https://github.com/mabrasil/codeicons}{Code
Icons}}{Code Icons}}\label{code-icons}

\begin{quote}
Code Icons is an icon set of 138 programming languages, frameworks, and
coding tools.
\end{quote}

\paragraph{\texorpdfstring{\href{https://github.com/mabrasil/milla-theme}{Milla
Theme}}{Milla Theme}}\label{milla-theme}

\begin{quote}
A simple \emph{(and alternative)} dark theme for SublimeText, Textmate
and other text editors that uses a flat color palette.
\end{quote}

\subsection{Extras}\label{extras}

\subsubsection{Speaker at:}\label{speaker-at}

\paragraph{\texorpdfstring{\href{http://flisolce.org/}{FLISOL
2015}}{FLISOL 2015}}\label{flisol-2015}

\begin{itemize}
\tightlist
\item
  \textbf{Title:} \emph{Uma Introdução a Git \& Github}
\item
  \textbf{When:} April 25, 2015
\item
  \textbf{Where:} Instituto Federal de Educação, Ciência e Tecnologia do
  Ceará, Campus Fortaleza.
\item
  \textbf{Attendees:} ≈60
\end{itemize}

\paragraph{\texorpdfstring{\href{http://www.cocoaheads.com.br/agendas/detalhes/79/}{3º
CocoaHeads
Fortaleza}}{3º CocoaHeads Fortaleza}}\label{uxba-cocoaheads-fortaleza}

\begin{itemize}
\tightlist
\item
  \textbf{Title:} \emph{Aplicações Híbridas com Ionic}
\item
  \textbf{When:} May 28, 2015
\item
  \textbf{Where:} Instituto Federal de Educação, Ciência e Tecnologia do
  Ceará, Campus Fortaleza.
\item
  \textbf{Attendees:} ≈40
\end{itemize}

\paragraph{\texorpdfstring{\href{http://pylestras.org/evento/ix-pylestras/}{IX
Pylestras}}{IX Pylestras}}\label{ix-pylestras}

\begin{itemize}
\tightlist
\item
  \textbf{Title:} \emph{Aventurando-se com metadados: Micro Data e Open
  Graph}
\item
  \textbf{When:} May 30, 2015
\item
  \textbf{Where:} Universidade de Fortaleza - Unifor.
\item
  \textbf{Attendees:} ≈40
\end{itemize}

\paragraph{\texorpdfstring{\href{http://www.meetup.com/pt/Ionic-Ceara/events/224620543/}{I
Ionic Meetup
Fortaleza}}{I Ionic Meetup Fortaleza}}\label{i-ionic-meetup-fortaleza}

\begin{itemize}
\tightlist
\item
  \textbf{Title:} \emph{Um projeto mobile open source em duas semanas
  (ou uma talk sobre como melhorar seu workflow Ionic)}
\item
  \textbf{When:} August 15, 2015
\item
  \textbf{Where:} Universidade de Fortaleza - Unifor.
\item
  \textbf{Attendees:} ≈40
\end{itemize}

\paragraph{\texorpdfstring{\href{http://unidevce.github.io/}{Unidev
2}}{Unidev 2}}\label{unidev-2}

\begin{itemize}
\tightlist
\item
  \textbf{Title:} \emph{De volta para o futuro: funcional nos dias
  atuais}
\item
  \textbf{When:} September 26, 2015
\item
  \textbf{Where:} Universidade de Fortaleza - Unifor.
\item
  \textbf{Attendees:} ≈50
\end{itemize}

\paragraph{\texorpdfstring{\href{http://www.seti.ufc.br/}{SETIC
2015}}{SETIC 2015}}\label{setic-2015}

\begin{itemize}
\tightlist
\item
  \textbf{Title:} \emph{De volta para o futuro: funcional nos dias
  atuais}
\item
  \textbf{When:} October 20, 2015
\item
  \textbf{Where:} Universidade Federal do Ceará - Campus do Pici.
\item
  \textbf{Attendees:} ≈30
\end{itemize}

\subsubsection{Organizer at:}\label{organizer-at}

\begin{itemize}
\tightlist
\item
  \href{http://ionicbrazil.com/}{Ionic Meetup Fortaleza}
\item
  \href{https://github.com/lambda-io}{Lambda I/O Foundation}
\item
  \href{http://js4girls-fortaleza.github.io/}{JS4Girls Fortaleza}
\end{itemize}

\end{document}
